\documentclass{article}
\usepackage{graphicx}
\graphicspath{{images/}}
\usepackage{titling}
\newcommand{\subtitle}[1]{%
  \posttitle{%
    \par\end{center}
    \begin{center}\large#1\end{center}
    \vskip0.5em}%
}
\begin{document}
\title{mbeddr Header Importer Architecture}
\subtitle{How Header Importer Works?}
\author{Mohammadreza Basirati}
\maketitle

\section{Introduction}
This document gives a brief but complete description about how the tool Header Importer works. It assumes that you are familiar with lexical scanner, lexical parser and Jetbrains MPS\footnote{More information about MPS on https://www.jetbrains.com/mps/}. 

Header Importer is a tool to import declarations from C header files into an mbeddr\footnote{More information about mbeddr on http://mbeddr.com/} project, so developers can use them in their code.

The process of importing a header file into an mbeddr project goes in three steps: 1. Scanning header files 2. Parsing scanner tokens 3. Importing declarations into mbeddr external module structure. We will discuss each step in detail in the following sections.

\begin{figure}
\caption{Header Importer Architecture}
\includegraphics[scale=0.55]{headerimporterdiagram.jpg}
\end{figure}

Step one and two work along in a Java project to prepare the required input for step three. The header importer tool Java project is located in folder bparser. Our scanner generator produces lexer.java file and our parser generator produces two files: sym.java and parser.java. Lexer.java will tokenize the input file. Parser.java file can recognize the tokens by their identifier which has been defined in sym.java file. The last part of the Java project consists of classes which will be used to keep header file declarations. At the last stage, MPS importer get the declarations and import them into an mbeddr external module(Figure 1).

\section{Scanner}
The first stage is to tokenize the header file. For this task we used JFlex\footnote{JFlex Web Site http://jflex.de/} scanner. You can find the scanner files for the header importer inside the \texttt{"scanner\_parser"} folder. For compatibility issue of our tool over different versions of header files, we enhanced the scanner to be able to tokenize the gcc stdio.

The scanner has three states which can identify one line comments, multiple lines comments, and strings. All the charachter sequences tokens that the scanner returns consists of: all possible operators in c, define, undef, if, ifdef, ifndef, else, endif, include, extern, pragma, typedef, struct, all c types (int, char, etc.), numbers , and identifiers. Right now, the parser does not use all operators' tokens, but they can be used in future for possible improvement.

The scanner can recognize the compiler preprocessing words which begin with double underscore and returns the token COMPWORD for them. Other compiler preprocessing phrases will be ignored by the scanner.

The lines that begin with define keyword for declaring macros and constants, scanner returns the whole line to the parser and doesn't tokenize their expression. In the parser section we will discuss this issue in more detail.


\section{Parser}
The parser has been generated by CUP\footnote{JFlex Web Site http://jflex.de/}. JFlex which we used for generating our scanner, is designed to work togetegher with CUP. Basically the parser gets tokens from the scanner and matches them by the grammers which we defined in our .cup file. The parser files of header importer located in \texttt{"scanner\_parser"} folder.

The grammer of the parser consists of two main parts: preprocessing steps and general declarations. Preprocessing steps comprise of the define, if, ifdef, ifndef, else, endif, include, and COMPWORD token that is the token returned by scanner for preprocessing words used by compilers.

General declaration part, consists of 4 type of declaration: typedef declaration, struct declaration, variable declaration, and function prototype declaration.

\subsection{Declaration Classes}
In the Java project we have several classes for managing and keeping the declarations. These classes are used by parser to define declarations and used by MPS importer part to get the declarations and import them into mbeddr module. In this section we will have a description for each class and how they colaborate in the parsing phase.

\begin{itemize}
\item[CodeGenerator]
\item[Declaration]
\item[ConditionalBlock]
\item[Define]
\item[Function]
\item[Variable]
\item[Struct]
\item[Typedef]
\item[Include]
\end{itemize}

\section{MPS Importer}

\end{document}
